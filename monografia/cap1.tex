% ----------------------------------------------------------
% Introdução
% ----------------------------------------------------------
\chapter{\textbf{INTRODUÇÃO}} 

Pessoas com deficiência visual, seja parcial ou total, enfrentam uma série de desafios em suas atividades cotidianas. Segundo dados do Instituto Brasileiro de Geografia e Estatística (IBGE), há mais de 6,5 milhões de pessoas com deficiência visual no Brasil, das quais cerca de 500 mil são cegas e 6 milhões possuem baixa visão \cite{Univali2024}. A escassez de recursos acessíveis, como o uso ampliado do sistema Braille, a audiodescrição e o desenvolvimento de tecnologias assistivas, dificulta ou limita o acesso à informação, à mobilidade e à autonomia pessoal. A locomoção em ambientes desconhecidos, por exemplo, é prejudicada pela ausência de sinalização adequada e pela insuficiência de rotas acessíveis, comprometendo a segurança e independência dos usuários. Tarefas básicas, como cozinhar, limpar ou localizar objetos, também se tornam complexas devido à falta de dispositivos adaptados às necessidades visuais \cite{alexandrino2017, depaula2008, pintanel2013}. 

Diante desse cenário, torna-se fundamental o desenvolvimento de soluções inovadoras e acessíveis que possam mitigar essas barreiras e promover a inclusão social de pessoas com deficiência visual. Tecnologias assistivas voltadas à orientação e mobilidade despontam como ferramentas promissoras para garantir maior independência e qualidade de vida, ao permitir a exploração segura e autônoma de diferentes espaços, inclusive no ambiente universitário \cite{depaula2008}.

Nas últimas décadas, tem-se observado uma evolução significativa nas tecnologias assistivas, com a incorporação de recursos baseados em áudio, como sintetizadores de voz e audiolivros, e, mais recentemente, com o uso de inteligência artificial e visão computacional. Esses avanços têm permitido a criação de ferramentas mais sofisticadas, capazes de reconhecer ambientes, objetos e padrões visuais com alto grau de precisão, adaptando-se às necessidades específicas dos usuários \cite{soares2017}.

O impacto do desenvolvimento de soluções dessa natureza extrapola o nível individual, ao contribuir para uma sociedade mais acessível, inclusiva e equitativa. Tecnologias de assistência visual podem facilitar a inserção de pessoas com deficiência em atividades educacionais, profissionais e sociais, promovendo a diversidade e o exercício pleno da cidadania \cite{agarwal2021-learning_representations, depaula2008}.

Paralelamente, a área de visão computacional tem experimentado avanços expressivos, com a aplicação de algoritmos de aprendizado profundo para reconhecimento de padrões visuais complexos. Algoritmos como o \textit{You Only Look Once} (YOLO), amplamente utilizados em sistemas de detecção em tempo real, têm se mostrado eficazes na identificação de objetos diversos em vídeos e imagens. Contudo, desafios técnicos ainda persistem, especialmente em cenários externos e não controlados, nos quais fatores como variações de iluminação, oclusões e ruído visual podem comprometer a robustez dos modelos \cite{gautam2021}.

Neste contexto, o presente trabalho desenvolve um sistema de reconhecimento de objetos voltado à assistência de pessoas com deficiência visual, utilizando algoritmos de visão computacional. A proposta é aplicar um modelo da família YOLO para detectar objetos de interesse, como bancos, faixas de pedestres e placas de ponto de ônibus, em vídeos gravados no campus Pampulha da Universidade Federal de Minas Gerais (UFMG) que simulam o algoritmo implementado em um celular. A detecção é convertida em alertas auditivos que indicam a presença e a localização relativa desses objetos, com o objetivo de oferecer uma experiência de navegação mais segura, autônoma e acessível.

\section{\textbf{Objetivo Geral}}

Desenvolver um sistema de reconhecimento de objetos e elementos de interesse no campus Pampulha da UFMG, utilizando algoritmos de visão computacional, com o objetivo de fornecer assistência à locomoção de pessoas com deficiência visual por meio de alertas auditivos.

\section{\textbf{Objetivos Específicos}}

\begin{itemize}
\item	Construir um banco de dados de objetos de interesse em diferentes condições ambientais e de iluminação dentro do campus Pampulha da UFMG;
\item	Preparar o conjunto de dados utilizando ferramentas de rotulagem e técnicas de \textit{data augmentation} para ampliar a robustez do modelo;
\item	Treinar e validar um modelo da família YOLO, comparando versões e avaliando métricas como \textit{precision}, \textit{recall} e \textit{mean average precision} (mAP) para diferentes classes de objetos;
\item	Desenvolver uma programação capaz de processar vídeos e emitir alertas auditivos dinâmicos com base na posição dos objetos detectados na cena;
\item   Avaliar qualitativamente e quantitativamente a eficácia do sistema por meio da análise de métricas e inspeção visual dos resultados em diferentes condições de teste.

\end{itemize}
